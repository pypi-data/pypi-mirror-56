\documentclass[10pt,reqno]{amsart}
\usepackage[left=.5in,right=.5in,top=.5in,bottom=.5in]{geometry}

\usepackage{booktabs}
\usepackage{amssymb}
\usepackage{graphicx}
\usepackage{epstopdf}
\usepackage{url}
\usepackage[usenames,dvipsnames,svgnames,table]{xcolor}

\usepackage{pgf}

\def\visible<#1>{}  % beamer command not needed here

\usepackage[utf8]{inputenc}

\usepackage[english]{babel}
\usepackage{amsfonts}
\usepackage{amsmath}
\usepackage{latexsym}
\usepackage{subfigure}
\usepackage{enumerate}

\usepackage{hyperref}  

\usepackage{ifpdf}

%Math Operators
\DeclareMathOperator    \aff                    {aff}
\DeclareMathOperator    \argmin         {arg\,min}
\DeclareMathOperator    \argmax         {arg\,max}
\DeclareMathOperator    \bd                     {bd}
\DeclareMathOperator    \cl                     {cl}
\DeclareMathOperator    \conv           {conv}
\DeclareMathOperator    \cone           {cone}
\DeclareMathOperator    \dist           {dist}
\DeclareMathOperator    \ep                     {exp}
\DeclareMathOperator    \et                     {ext}
\DeclareMathOperator    \ext                    {ext}
\DeclareMathOperator    \intr                   {int}
\DeclareMathOperator    \lin                    {lin}
\DeclareMathOperator    \proj           {proj}
\DeclareMathOperator    \rec                    {rec}
\DeclareMathOperator    \rk                     {rk}
\DeclareMathOperator    \relint         {rel\,int}
\DeclareMathOperator    \spann          {span}
\DeclareMathOperator    \verts          {vert}
\DeclareMathOperator    \vol                    {vol}
\DeclareMathOperator    \Aff {Aff}  % Grp of invertible affine linear transformations


%Mathbb
\newcommand{\R}{\mathbb R}
\newcommand{\Q}{\mathbb Q}
\newcommand{\Z}{\mathbb Z}
\newcommand{\N}{\mathbb N}
\newcommand{\C}{\mathbb C}
\newcommand{\T}{\mathbb T}
\renewcommand{\r}{\bar{r}}
%\newcommand{\p}{\bar{p}}

%\newcommand{\verts}{\mathrm{vert}}
%\renewcommand{\intr}{\mathrm{int}}

%Constructed Commands
\newcommand{\floor}[1]{\lfloor#1\rfloor}
\newcommand{\ceil}[1]{\lceil #1 \rceil}
\newcommand\st{\mid}
\newcommand\bigst{\mathrel{\big|}}
\newcommand\Bigst{\mathrel{\Big|}}

%% Vectors
\def\ve#1{\mathchoice{\mbox{\boldmath$\displaystyle\bf#1$}}
{\mbox{\boldmath$\textstyle\bf#1$}}
{\mbox{\boldmath$\scriptstyle\bf#1$}}
{\mbox{\boldmath$\scriptscriptstyle\bf#1$}}}

%%%%%%%%%%%%%%%%%%%%%%%%%

%          SPECIFIC COMMANDS FOR THIS PAPER        %

%%%%%%%%%%%%%%%%%%%%%%%%%

% Random new commands
\newcommand{\bpi}{\bar \pi}
\newcommand{\setcond}[2]{\left\{\, #1 : #2 \,\right\}}
% Mathcal 
\renewcommand{\P}{\mathcal{P}}
\newcommand{\D}{\mathcal{D}}
\renewcommand{\S}{\mathcal{S}}
\newcommand{\Stri}{\S_{q,\tri}}
\newcommand{\barStri}{\bar\S_{q,\tri}}

\newcommand{\E}{\mathcal{E}}
\newcommand{\G}{\mathcal{G}}

%
%% Bold face letters
\newcommand{\rx}{{\ve r}}
\newcommand{\x}{{\ve x}}
\newcommand{\y}{{\ve y}}
\newcommand{\z}{{\ve z}}
\renewcommand{\v}{{\ve v}}
\newcommand{\g}{{\ve g}}
\newcommand{\e}{{\ve e}}
\renewcommand{\u}{{\ve u}}
\renewcommand{\a}{{\ve a}}
\newcommand{\f}{{\ve f}}
\newcommand{\0}{{\ve 0}}
\newcommand{\m}{{\ve m}}
\newcommand{\p}{{\ve p}}
\renewcommand{\t}{{\ve t}}
\newcommand{\w}{{\ve w}}
\renewcommand{\b}{{\ve b}}
\renewcommand{\d}{{\ve d}}
\newcommand{\cve}{{\ve c}}
\newcommand{\h}{{\ve h}}
\newcommand{\rr}{{\ve r}}
\newcommand{\gp}{{\ve {\bar g}}}
\newcommand{\gt}{{\ve {\tilde g}}}
\newcommand{\gs}{{\ve  g}}

\newcommand{\Ball}{B}
% No longer mathcal
\newcommand{\B}{B}

\def\st{\mid}
\newenvironment{psmallmatrix}{\left(\smallmatrix}{\endsmallmatrix\right)}
\newenvironment{psmallmatrixbig}{\bigl(\smallmatrix}{\endsmallmatrix\bigr)}

%%% Redefine \pmod so that it respects \textstyle.
\makeatletter
\renewcommand{\pod}[1]%% a helper command used in \pmod.
{\allowbreak\mathchoice{\mkern18mu}{\mkern8mu}{\mkern8mu}{\mkern8mu}(#1)}
\makeatother

\chardef\Myunderscore=`\_
%\let\Myunderscore=\textunderscore
% from http://texwelt.de/wissen/fragen/565/was-heit-hyperrefs-warnung-token-not-allowed-in-a-pdf-string
\pdfstringdefDisableCommands{%
  \def\Myunderscore{\textunderscore}%
}
\newcommand\underscore{\Myunderscore\allowbreak}


\newcommand\githubsearchurl{https://github.com/mkoeppe/infinite-group-relaxation-code/search}
\input{../survey_graphics/sage-commands}
\DeclareRobustCommand\sage[1]{\texttt{#1}}
\DeclareRobustCommand\sagefunc[1]{\pgfkeys{/sagefunc/#1}}

\usepackage[normalem]{ulem}

\usepackage{verbatim}

  \newcommand\CompendiumGraphics[1]{\includegraphics[width=.8\linewidth]{#1}}
  \newcommand\CompendiumGridEntry[1]{\CompendiumGraphics{#1}\par{\tiny\sagefunc{#1}\par}\vspace*{-3ex}}

\title[Infinite Group Problem Code: figures test suite]{Infinite Group Problem Code:\\figures test suite}

\begin{document}

\setcounter{section}{1}

\section{Figures from \itshape An electronic compendium
  of extreme functions for the
  Gomory--Johnson
  infinite group problem}

\begin{figure}[htbp]
  \centering
  \includegraphics[width=0.6\linewidth]{rlm_dpl1_extreme_3a-2d_diagram.pdf}
  \caption{Diagram of the function \sagefunc{rlm_dpl1_extreme_3a} (\emph{blue
      graphs on the top and the left}) 
    and its polyhedral complex $\Delta\P$ (\emph{gray
      solid lines}).
    The set $E(\pi)$ is the union of the faces shaded in green.
    The \emph{heavy diagonal green line} $x + y = 1+f$ 
    corresponds to the symmetry condition (the line $x+y = f$ appears as an
    edge of $F_1$).  Vertices of $\Delta\P$ do not
    necessarily project (\emph{dotted gray lines}) to breakpoints.  At the
    borders, the projections $p_i(F)$ of two-dimensional additive faces are
    shown as \emph{gray shadows}: $p_1(F)$ at the top border, $p_2(F)$ at the
    left border, $p_3(F)$ at the bottom and the right borders.}
  \label{fig:rlm_dpl1_extreme_3a}
\end{figure}

\begin{figure}[htbp]
  \centering
  \includegraphics[width=0.45\linewidth]{chen_3_slope_not_extreme-covered_intervals}\quad
  \includegraphics[width=0.45\linewidth]{chen_3_slope_not_extreme-perturbation-1}
  \caption{The function \sagefunc{chen_3_slope_not_extreme} is 
    minimal, but not extreme, as proved by
    \sage{\sagefunc{extremality_test}(h, show\underscore{}plots=True)}.
    The procedure first shows that
    for any distinct minimal $\pi^1 = \pi + \bar\pi$ (\emph{blue}), $\pi^2 = \pi
    - \bar\pi$ (\emph{red}) such that $\pi = \tfrac{1}{2}\pi^1
    + \tfrac{1}{2} \pi^2$, the functions $\pi^1$ and $\pi^2$ are continuous
    piecewise linear with the same breakpoints as $\pi$.  A finite-dimensional extremality
    test then finds a perturbation~$\bar\pi$ (\emph{magenta}), as shown.}
  \label{fig:chen_3_slope_not_extreme}
\end{figure}

\begin{figure}[htbp]
  \centering
  \includegraphics[width=0.6\linewidth]{drlm_backward_3_slope-2d_diagram.pdf}
  \caption{The \sagefunc{drlm_backward_3_slope} function}
  \label{fig:drlm_backward_3_slope}
\end{figure}

\begin{figure}[htbp]
  \centering
  \includegraphics[width=0.8\linewidth]{kf_n_step_mir-lemma5_1.pdf}
  \caption{The \sagefunc{kf_n_step_mir} function}
  \label{fig:kf_n_step_mir}
\end{figure}

\end{document}

%%% Local Variables:
%%% mode: latex
%%% TeX-master: t
%%% End:

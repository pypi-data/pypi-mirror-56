\documentclass[10pt,reqno]{amsart}
%\documentclass[runningheads,a4paper,envcountsame,envcountsect]{llncs}
%\usepackage{pgfmath,pgffor}

\usepackage{pdfpages}
%% \usepackage{longtable} % for 'longtable' environment
\usepackage{pdflscape} % for 'landscape' environment

\usepackage{booktabs}
\usepackage{amssymb}
%\setcounter{tocdepth}{3}  %% causes errors?! --Matthias
\usepackage{graphicx}
\usepackage{epstopdf}
%%\usepackage[headings]{fullpage}
\usepackage[hyphens]{url}
%% \newcommand{\keywords}[1]{\par\addvspace\baselineskip
%% \noindent\keywordname\enspace\ignorespaces#1}
%\usepackage{color}
%\usepackage[usenames,svgnames,table]{xcolor}
%\usepackage{equipert}
\usepackage[square,numbers,sort&compress]{natbib}

\usepackage{multirow}

\usepackage{tikz}
\definecolor{mediumspringgreen}{rgb}{0.0, 0.98039215, 0.60392156}

\usepackage{pgfplots} %% This Clashes with other things

\usetikzlibrary{positioning}


\def\visible<#1>{}  % beamer command not needed here

\usepackage[utf8]{inputenc}
% \usepackage{xr}

\usepackage[english]{babel}
\usepackage{amsfonts}
\usepackage{amsmath}
\usepackage{latexsym}
\usepackage{color}
\usepackage{subfigure}
\usepackage{enumerate}

\usepackage{hyperref}  

\usepackage{ifpdf}
%\usepackage{verbatim}
\usepackage{listings}
\lstset{
  %basicstyle=\tiny\ttfamily,
  basicstyle=\tiny\sffamily,
  breaklines=true,
  breakatwhitespace=false,
  columns=fullflexible,
  breakindent=1em
}

\newcommand\inputfig[1]{\ifpdf
\input{#1.pdftex_t}% \input{#1.pdf_t}
\else
\input{#1.pstex_t}
\fi}
\newcommand\inputfigeps[1]{\ifpdf
\includegraphics{#1.pdf}% \input{#1.pdf_t}
\else
\includegraphics{#1.eps}
\fi
}

%Math Operators
\DeclareMathOperator    \aff                    {aff}
\DeclareMathOperator    \argmin         {arg\,min}
\DeclareMathOperator    \argmax         {arg\,max}
\DeclareMathOperator    \bd                     {bd}
\DeclareMathOperator    \cl                     {cl}
\DeclareMathOperator    \conv           {conv}
\DeclareMathOperator    \cone           {cone}
\DeclareMathOperator    \dist           {dist}
\DeclareMathOperator    \ep                     {exp}
\DeclareMathOperator    \et                     {ext}
\DeclareMathOperator    \ext                    {ext}
\DeclareMathOperator    \intr                   {int}
\DeclareMathOperator    \lin                    {lin}
\DeclareMathOperator    \proj           {proj}
\DeclareMathOperator    \rec                    {rec}
\DeclareMathOperator    \rk                     {rk}
\DeclareMathOperator    \relint         {rel\,int}
\DeclareMathOperator    \spann          {span}
\DeclareMathOperator    \verts          {vert}
\DeclareMathOperator    \vol                    {vol}
\DeclareMathOperator    \Aff {Aff}  % Grp of invertible affine linear transformations


\newcommand{\vect}[1]{\left(\begin{array}{@{}c@{}}#1\end{array}\right)}


%\newcommand{\mod}{\mathrm{mod}\;}

% to blank out text:
\newcommand{\old}[1]{{}}


%Command Shortcuts
\newcommand{\ol}{\overline}
\newcommand{\sm}{\setminus}
\newcommand{\bb}{\mathbb}


%Mathbb
\newcommand{\R}{\bb R}
\newcommand{\Q}{\bb Q}
\newcommand{\Z}{\bb Z}
\newcommand{\N}{\bb N}
\newcommand{\C}{\bb C}
\newcommand{\T}{\mathcal T}
\renewcommand{\r}{\bar{r}}
%\newcommand{\p}{\bar{p}}

%\newcommand{\verts}{\mathrm{vert}}
%\renewcommand{\intr}{\mathrm{int}}

%Constructed Commands
\newcommand{\floor}[1]{\lfloor#1\rfloor}
\newcommand{\ceil}[1]{\lceil #1 \rceil}
\newcommand\st{\mid}
\newcommand\bigst{\mathrel{\big|}}
\newcommand\Bigst{\mathrel{\Big|}}

%% Vectors
\def\ve#1{\mathchoice{\mbox{\boldmath$\displaystyle\bf#1$}}
{\mbox{\boldmath$\textstyle\bf#1$}}
{\mbox{\boldmath$\scriptstyle\bf#1$}}
{\mbox{\boldmath$\scriptscriptstyle\bf#1$}}}

%% %%% Proofs with qeds

%% \let\proofqed=\qed
%% \newcommand\qedhere{\qed\global\let\proofqed=\relax}

%% \let\saveproof=\proof
%% \def\proof{\saveproof\global\let\proofqed=\qed}
%% \let\saveendproof=\endproof
%% \def\endproof{\proofqed\saveendproof}


%%%%%%%%%%%%%%%%%%%%%%%%%

%          SPECIFIC COMMANDS FOR THIS PAPER        %

%%%%%%%%%%%%%%%%%%%%%%%%%

% Random new commands
\newcommand{\bpi}{\bar \pi}
\newcommand{\varphiD}{\psi_{q,\diag}}
\newcommand{\psiPoint}{\psi^m_{q,\point}}
\newcommand{\setcond}[2]{\left\{ #1 \,\st\, #2 \right\}}
% Mathcal 
\newcommand{\I}{\mathcal{P}}  %%NOTE THE CHANGE!!!
% \newcommand{\diag}{\smallsetminus}
\newcommand{\Itri}[1][q]{\I_{#1,\tri}}
\newcommand{\Idiag}[1][q]{\I_{#1,\diag}}
\newcommand{\Ivert}[1][q]{\I_{#1,\ver}}
\newcommand{\Iverthor}[1][q]{\I_{#1,\ver\,\hor}}
\newcommand{\Ihor}[1][q]{\I_{#1,\hor}}
\newcommand{\Iedge}[1][q]{\I_{#1,\edge}}
\newcommand{\Ipoint}[1][q]{\I_{#1,\point}}
\newcommand{\Ipointedge}[1][q]{\I_{#1,\point\,\edge}}
\newcommand{\Ipointdiag}[1][q]{\I_{#1,\point\,\diag}}
\newcommand{\Ipointdiagtri}[1][q]{\I_{#1,\point\,\diag\,\tri}}
\renewcommand{\P}{\mathcal{P}}
\newcommand{\D}{\mathcal{D}}
\renewcommand{\S}{\mathcal{S}}
\newcommand{\Stri}{\S_{q,\tri}}
\newcommand{\barStri}{\bar\S_{q,\tri}}
%%%% Notation updates:

\newcommand{\E}{\mathcal{E}}
\newcommand{\G}{\mathcal{G}}

\newcommand\EqClass[1]{[#1]} % Equivalence class of a triangle modulo integer
                             % translation. 

%
%% Bold face letters
\newcommand{\rx}{{\ve r}}
\newcommand{\x}{{\ve x}}
\newcommand{\y}{{\ve y}}
\newcommand{\z}{{\ve z}}
\renewcommand{\v}{{\ve v}}
\newcommand{\g}{{\ve g}}
\newcommand{\e}{{\ve e}}
\renewcommand{\u}{{\ve u}}
\renewcommand{\a}{{\ve a}}
\newcommand{\f}{{\ve f}}
\newcommand{\0}{{\ve 0}}
\newcommand{\m}{{\ve m}}
\newcommand{\p}{{\ve p}}
\renewcommand{\t}{{\ve t}}
\newcommand{\w}{{\ve w}}
\renewcommand{\b}{{\ve b}}
\renewcommand{\d}{{\ve d}}
\newcommand{\cve}{{\ve c}}
\newcommand{\h}{{\ve h}}
\newcommand{\rr}{{\ve r}}
\newcommand{\gp}{{\ve {\bar g}}}
\newcommand{\gt}{{\ve {\tilde g}}}
\newcommand{\gs}{{\ve  g}}

\newcommand{\ie}{ i.e. }

\newcommand{\Ball}{B}
% No longer mathcal
\newcommand{\B}{B}

\def\st{\mid}
\newenvironment{psmallmatrix}{\left(\smallmatrix}{\endsmallmatrix\right)}
\newenvironment{psmallmatrixbig}{\bigl(\smallmatrix}{\endsmallmatrix\bigr)}
\newcommand\InlineFrac[2]{#1/#2}  % only works for simple arguments
\newcommand\ColVec[3][\relax]% Optional argument is denominator.
{
  \ifx#1\relax
  % no denominator
  \bgroup\let\frac=\InlineFrac\begin{psmallmatrixbig}#2\vphantom{/}\\#3\vphantom{/}\end{psmallmatrixbig}\egroup
  \else
  %\frac1{#1}\bgroup\let\frac=\InlineFrac\begin{psmallmatrix}#2\\#3\end{psmallmatrix}\egroup
  \bgroup\let\frac=\InlineFrac\begin{psmallmatrixbig}\ifx#200\else#2/#1\fi\\\ifx#300\else#3/#1\fi\end{psmallmatrixbig}\egroup
  \fi
}

%Theorem Environments
\newtheorem{theorem}{Theorem}[section]

%% create new theorem numbered just as Theorem, but which works with autoref.
\makeatletter
\newcommand\MkNewTheorem[2]{%
  \newtheorem{#1}{#2}
  \expandafter\def\csname c@#1\endcsname{\c@theorem}
  \expandafter\def\csname p@#1\endcsname{\p@theorem}
  \expandafter\def\csname the#1\endcsname{\thetheorem}
  \expandafter\def\csname #1name\endcsname{#2}
}

\MkNewTheorem{corollary}{Corollary}
\MkNewTheorem{lemma}{Lemma}
\MkNewTheorem{proposition}{Proposition}
\MkNewTheorem{prop}{Proposition}
\MkNewTheorem{claim}{Claim}
\MkNewTheorem{observation}{Observation}
\MkNewTheorem{obs}{Observation}
\MkNewTheorem{conjecture}{Conjecture}
\MkNewTheorem{openquestion}{Open question}
\MkNewTheorem{question}{Question}

\theoremstyle{definition}
\MkNewTheorem{example}{Example}
\MkNewTheorem{exercise}{Exercise}
\MkNewTheorem{notation}{Notation}
\MkNewTheorem{assumption}{Assumption}
\MkNewTheorem{definition}{Definition}
\MkNewTheorem{remark}{Remark}
\MkNewTheorem{goal}{Goal}
\MkNewTheorem{problem}{Problem}

% \newtheorem{obs}[theorem]{Observation}
% \newtheorem{prop}[theorem]{Proposition}
% \newtheorem{assumption}{Assumption}

%%% We want boldmath everywhere bold appears
\makeatletter
\let\OurMathBbAux=\mathbb
\DeclareRobustCommand\OurMathBb{\OurMathBbAux}
\let\mathbb=\OurMathBb
\let\bfseries=\undefined
\DeclareRobustCommand\bfseries
{\not@math@alphabet\bfseries\mathbf
  \boldmath\fontseries\bfdefault\selectfont\let\OurMathBbAux=\mathbf}
\def\@thm#1#2#3{%
  \ifhmode\unskip\unskip\par\fi
  \normalfont
  \trivlist
  \let\thmheadnl\relax
  \let\thm@swap\@gobble
  \thm@notefont{\fontseries\mddefault\upshape\unboldmath}%   %%% <--- Added \unboldmath
  \thm@headpunct{.}% add period after heading
  \thm@headsep 5\p@ plus\p@ minus\p@\relax
  \thm@space@setup
  #1% style overrides
  \@topsep \thm@preskip               % used by thm head
  \@topsepadd \thm@postskip           % used by \@endparenv
  \def\@tempa{#2}\ifx\@empty\@tempa
    \def\@tempa{\@oparg{\@begintheorem{#3}{}}[]}%
  \else
    \refstepcounter{#2}%
    \def\@tempa{\@oparg{\@begintheorem{#3}{\csname the#2\endcsname}}[]}%
  \fi
  \@tempa
}
\makeatother

%%% Redefine \pmod so that it respects \textstyle.
\makeatletter
\renewcommand{\pod}[1]%% a helper command used in \pmod.
{\allowbreak\mathchoice{\mkern18mu}{\mkern8mu}{\mkern8mu}{\mkern8mu}(#1)}
\makeatother

%% We certainly do not want to distinguish the two:
\let\epsilon=\varepsilon

%\chardef\Myunderscore=`\_
\let\Myunderscore=\textunderscore   %%%% Better with textsf
% from http://texwelt.de/wissen/fragen/565/was-heit-hyperrefs-warnung-token-not-allowed-in-a-pdf-string
\pdfstringdefDisableCommands{%
  \def\Myunderscore{\textunderscore}%
}
\newcommand\underscore{\Myunderscore\allowbreak}
\let\_=\underscore

\newcommand\githubsearchurl{https://github.com/mkoeppe/infinite-group-relaxation-code/search}
%\input{sage-commands}
\DeclareRobustCommand\sage[1]{\textsf{\upshape #1}}
%\DeclareRobustCommand\sage[1]{\texttt{#1}}
\DeclareRobustCommand\sagefunc[1]{\pgfkeys{/sagefunc/#1}}
\DeclareRobustCommand\sagefuncgraph[1]{\raisebox{-0.08ex}{\includegraphics[height=2ex,width=2.5em]{funcgraphs/#1}}}
\DeclareRobustCommand\sagefuncwithgraph[1]{\sagefunc{#1} \sagefuncgraph{#1}}

\DeclareRobustCommand\sagefuncwithgraphgomoryfractional{\sagefunc{gomory_fractional}\ \smash{\raisebox{-0.08ex}{\includegraphics[height=2.65ex,width=2.5em]{funcgraphs/gomory_fractional}}}}


%% \title[Two-sided discontinuous piecewise linear minimal valid functions]{Two-sided discontinuous piecewise linear minimal valid functions 
%%   that are not extreme,
%%   but which are not convex combinations of other piecewise linear minimal valid functions
%% }

% \thanks{The authors gratefully acknowledge partial support from the National Science
%   Foundation through grant DMS-1320051, awarded to M.~K\"oppe.}

%% \author{Matthias K\"oppe}
%% \address{Matthias K\"oppe: Dept.\ of Mathematics, University of California, Davis}
%% \email{mkoeppe@math.ucdavis.edu}

%% \author{Yuan Zhou} 
%% \address{Yuan Zhou: Dept.\ of Mathematics, University of Kentucky}
%% \email{yuan.zhou@uky.edu}
%% 719 Patterson Office Tower
%% Lexington, Kentucky 40506-0027



\date{$\relax$Revision: 2348 $ - \ $Date: 2018-01-28 20:35:05 -0800 (Sun, 28 Jan 2018) $ $\!\!\!}

\usepackage[normalem]{ulem}

%\graphicspath{{../reu-2013/}}
\newcommand\Figure[2][\relax]{%
  \begin{figure}[h!]
    \includegraphics[width=.8\textwidth]{#2}
    \caption{\ifx#1\relax#2\else#1\fi}
  \end{figure}
}

\begin{document}
 \newcommand{\tgreen}[1]{\textsf{\textcolor {ForestGreen} {#1}}}
 \newcommand{\tred}[1]{\texttt{\textcolor {red} {#1}}}
 \newcommand{\tblue}[1]{\textcolor {blue} {#1}}

%\newcommand{\tgreen}[1]{#1}
%\newcommand{\tred}[1]{}
%\newcommand{\tblue}[1]{#1}


%\clearpage
%% {\footnotesize
%% \tableofcontents}

\setcounter{page}{16}
\setcounter{section}{5}
\thispagestyle{plain}

\section{Figures from \itshape Equivariant Perturbation in Gomory and Johnson's Infinite Group 
  Problem. VI. The Curious Case of Two-Sided Discontinuous Minimal Valid
  Functions}

\begin{figure}[h]
%%%%%%%%%% Comment from survey: I decided that this figure, after all, is too distracting; it's
%%%%%%%%%% also a bit too complicated to explain.
\begin{center}
\includegraphics[width=.44\linewidth]{zhou_two_sided_discontinuous_cannot_assume_any_continuity-covered_intervals.png}\quad
\includegraphics[width=.44\linewidth]{zhou_two_sided_discontinuous_cannot_assume_any_continuity-perturbation-1.png}
\end{center}
\caption{This function, \sage{$\pi$ = \sage{zhou\_two\_sided\_discontinuous\_cannot\_assume\_any\_continuity}}, is
  minimal, but not extreme% (and  hence also not a facet)
  , as proved by
  \sage{extremality\_test($\pi$, show\underscore{}plots=True)}.
  The procedure first shows that
  for any distinct minimal $\pi^1 = \pi + \bar\pi$ (\emph{blue}), $\pi^2 = \pi
  - \bar\pi$ (\emph{red}) such that $\pi = \tfrac{1}{2}\pi^1
  + \tfrac{1}{2} \pi^2$, the functions $\pi^1$ and $\pi^2$ are
  piecewise linear with the same breakpoints as $\pi$ and possible additional
  breakpoints at~$\frac14$ and $\frac34$.   The open intervals between these
  breakpoints are covered%  (see \autoref{s:preliminaries}, after
  % \autoref{thm:directly_covered}, for this notion)
  . %  (in the terminology of
  % \cite{basu-hildebrand-koeppe:equivariant}, $\pi$ is \emph{affine imposing}
  % on all open intervals between these breakpoints
  %).  
  %% \tred{(TODO: Use plotting code from branch
  %% plot\_covered\_intervals\_with\_markers to mark the extra breakpoint.)}
  A finite-dimensional extremality
  test then finds exactly one linearly independent perturbation $\bar\pi$
  (\emph{magenta}), as shown. Thus all nontrivial perturbations are discontinuous at
  $\frac{3}{4}$, a point where $\pi$ is continuous.
}
\label{fig:two_sided_discontinuous_cannot_assume_any_continuity}
\end{figure}

\begin{figure}[h]
\centering
\includegraphics[width=\linewidth]{has_crazy_perturbation.png}
\caption{This function,
  \sage{$\pi$ = kzh\_minimal\_has\_only\_crazy\_perturbation\_1}, 
  % is the star of the present paper.  
  % It
  has three slopes (\emph{blue}, \emph{green}, \emph{red}) and is
  discontinuous on both sides of the origin. 
  It is a non-extreme minimal valid function, 
  but in order to demonstrate non-extremality, one needs to use a highly
  discontinuous (locally microperiodic) perturbation.
  We construct a simple explicit example perturbation $\epsilon\bar\pi$
  (\emph{magenta})% ; see \autoref{th:kzh_minimal_has_only_crazy_perturbation_1}
  .
  It takes three values, $\epsilon$, $0$, and $-\epsilon$ 
  (\emph{horizontal magenta line segments}) where $\epsilon=0.0003$; 
  in the figure it has been rescaled to amplitude~$\frac1{10}$.
}
\label{fig:has_crazy_perturbation}
\end{figure}

\def\COMMONSCALE{0.35}
\begin{figure}[hp]
\begin{minipage}{.48\textwidth}
\includegraphics[scale=\COMMONSCALE]{proof_uniform_cont_case_1.png}
\par\centering
  Case 1
\end{minipage}
\hfill
\begin{minipage}{.46\textwidth}
\includegraphics[scale=\COMMONSCALE]{proof_uniform_cont_case_3.png}
\par\centering
  Case 3
\end{minipage}
\vspace{3ex}\par
\centering
\begin{minipage}{.58\textwidth}
\includegraphics[scale=\COMMONSCALE]{proof_uniform_cont_case_2.png}
\par\centering
  Case 2
\end{minipage}
%\begin{minipage}{.32\textwidth}
%\includegraphics[width=\linewidth]{proof_uniform_cont_case_4.png}
%\end{minipage}
\caption{Illustration of the proof % of
  % \autoref{prop:near-pexider}
  for $U$. Three partial diagrams of 
  $\Delta\P$, where the tangent cone $C$ of a two-dimensional face $F\in
  \Delta\P$ at vertex $(u,v)$ is a (\textit{left}, Case~1): right-angle cone (first
  quadrant); 
  (\textit{bottom}, Case~2): obtuse-angle cone; (\textit{right}, Case~3):
  sharp-angle cone (contained in a second quadrant). The \emph{light green
    area} $C_\eta$ is contained in the face~$F$. The \emph{green sector} at
  $(u,v)$ indicates that 
  $\Delta\pi_F(u,v)=0$. The \emph{black points} inside the \emph{light green
    area} show the sequences used in the proof% of equation~\eqref{eq:case123-U}
  .}
\label{fig:proof_uniform_continuous}
\end{figure}

\begin{figure}[hp]
\begin{minipage}{.5\textwidth}
\includegraphics[scale=\COMMONSCALE]{proof_uniform_V_case12.png}
\par\centering
  Case 1 and Case 2
\end{minipage}
\hfill
\begin{minipage}{.4\textwidth}
\includegraphics[scale=\COMMONSCALE]{proof_uniform_V_case3.png}
\par\centering
  Case 3
\end{minipage}
\caption{Illustration of the proof % of
  % \autoref{prop:near-pexider}
  for~$V$% ,
  % equation~\eqref{eq:V-from-U}
  .}
\label{fig:V-from-U}
\end{figure}

\begin{figure}[hp]
\begin{minipage}{.4\textwidth}
\includegraphics[scale=\COMMONSCALE]{proof_uniform_W_case12.png}
\par\centering
  Case 1 and Case 2
\par\vspace{20ex}
\end{minipage}
\hfill
\begin{minipage}{.55\textwidth}
\includegraphics[scale=\COMMONSCALE]{proof_uniform_W_case3a.png}
\par{\centering
  Case 3a
\par\vspace{8ex}}
\includegraphics[scale=\COMMONSCALE]{proof_uniform_W_case3b.png}
\par\centering
  Case 3b
\end{minipage}
\caption{Illustration of the proof % of
  % \autoref{prop:near-pexider}
  for~$W$% ,
  % equation~\eqref{eq:W-from-U}.
}
\label{fig:W-from-U}
\end{figure}


\end{document}
%%% Local Variables:
%%% mode: latex
%%% TeX-master: t
%%% End:
